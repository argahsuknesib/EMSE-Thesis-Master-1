% Chapter Template

\chapter{Context} % Main chapter title

\label{ChapterX} % Change X to a consecutive number; for referencing this chapter elsewhere, use \ref{ChapterX}

%----------------------------------------------------------------------------------------
%	SECTION 1
%----------------------------------------------------------------------------------------

\section{Introduction}
The work is based on an Industry 4.0 scenario, which is a cyber-physical environment consisting of various different 
actors and objects involved. The different actors involved are either stationary or mobile. Moreover, complexity of the 
environment increases when we account for heterogeneous actors with various decision making capabilities. Robots with various
manufacturers present various transform frames, different software and sensors. Due to the heterogeneous nature of the robots 
involved, we can not depend on information we receive from the robot, as this particular information will differ from a robot 
to other based upon it's configuration. The problem is solved by creating a digital twin which records the information of the 
environment as well as the robot. The simulator notes the state of the robot and obstacles it surrounds as it passes 
through the obstacle grid. 
\subsection{Motivation}
The structure of a dynamic Industry 4.0 environment is highly volatile, the structure is defined through a stationary frame that 
has been declared before. The decision making capabilities of the robot to navigate the environment while avoiding obstacles and 
other robots can have a great impact on the performance and the utility of the environment. 
